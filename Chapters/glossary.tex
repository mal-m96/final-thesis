%!TEX root = ../template.tex
%%%%%%%%%%%%%%%%%%%%%%%%%%%%%%%%%%%%%%%%%%%%%%%%%%%%%%%%%%%%%%%%%%%%
%% glossary.tex
%% NOVA thesis document file
%%
%% Glossary definition
%%%%%%%%%%%%%%%%%%%%%%%%%%%%%%%%%%%%%%%%%%%%%%%%%%%%%%%%%%%%%%%%%%%%

\typeout{NT FILE glossary.tex}


\newglossaryentry{positive selection}{
	name={positive selection}, 
	description={also known as Darwin Selection, is the process by which new, more advantageous genetic variants sweep a population. It's the main mechanism that gives rise to evolution.}
}

\newglossaryentry{selection pressure}{
	name={selection pressure}, 
	description={are any kind of external agents which affect an organism's ability to survive in a given environment. Is often referred to as "Survival of the Fittest", organisms that survive are able to reproduce and pass on their favourable genes to their offspring leading to natural selection.}
}

\newglossaryentry{segregate}{
	name={segregating population}, 
	description={A group of organisms of the same species relatively isolated from other groups of the same species in which their alleles are randomly separated from each other.}
}
	
\newglossaryentry{SV}{
	name={Structural variation (SV)}, 
	description={is a rearrangement of part of the genome, approximately 1 kb (or larger) in size, that can include inversions and balanced translocations or genomic imbalances (insertions and deletions), commonly referred to as copy number variants (CNVs). They have been implicated in a number of conditions, from polycystic kidney disease, cardiomyopathies to, in some cases, causes  intellectual disability.}	
}
